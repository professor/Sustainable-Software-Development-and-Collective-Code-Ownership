

\begin{table*}[t]
\renewcommand{\arraystretch}{1.5}
\centering
\caption{Sample Quotes}
\label{ChainOfEvidenceTable}
\begin{tabular}{|p{1.65in}|p{5.55in}|}
\hline
\multicolumn{2}{|c|}{Underlying Principles} \\
\hline
\textbf{Keeping a Positive Attitude Toward Team Disruption} & \\
& \participantQuote{I'm excited when a new person joins the team. That person has experience that might add something to the project.} \\

& \participantQuote{I like that people bring new energy. Projects often get into the state of a lull with the same people working on it and have the same cadence. New people bring a new perspective. [Two engineers recently joined] and it was really cool to see their fresh perspective. I always like people joining a project.} \\

& \participantQuote{There's all this shared knowledge and context that is in our team culture. It is invisible to us. [Adding another person exposes it to the team.]}  \\
 & \participantQuote{You get an outsider's perspective. They bring a fresh set of eyes.} \\

& \participantQuote{The team is extremely welcoming. I was surprised to see that there was no \quotes{us} versus \quotes{them} between developers and product.}  \\

\hline
Encouraging Knowledge Sharing and Continuity & Example 1b \\
& \participantQuote{[When pairing], we discuss the change through, and what we want to do. We make decisions after our discussions. [At the end of the track], it's everybody's decision who has touched this code. It's really hard for me say that [any particular piece] was my idea. [Even still], I feel that I own that codebase.} \\

& \participantQuote{We managed to preserve [team code ownership] even though it's a big team and we had many people rotating on and off  the project.  We still had a team effort and the team owned the entire codebase.} \\

& \participantQuote{I have a general sense of confidence around the codebase. Given any story in the backlog, I'm reasonably confident that I could start it and figure it out.} \\

& \participantQuote{It is important to enable other programmers to quickly know the responsibility of each component and control flow.} \\

& \participantQuote{My ideal team size is two pairs. Anything that we did today, by tomorrow, the team will have an opportunity to know what we've done.}  After one rotation, context is shared immediately throughout the team. \\

& \participantQuote{Everyone has context about the entire system, and everyone's constantly communicating about the entire system.} \\

\hline
Caring about Code Quality & Example 1c \\

& \participantQuote{Everyone is responsible for fixing issues and everyone can fix it.} \\
& \participantQuote{We should encourage the feeling that since I can change anything, I need to make sure all the code I write and come across is good. } \\
& \participantQuote{It is important to create a culture of \quotes{it is ok to make a mistake.}} \\
& \participantQuote{I feel pretty good about the product and the code base.  We did a lot of good work and implemented many features.} \\
\hline
\end{tabular}
\end{table*}




\begin{table*}[t]
\renewcommand{\arraystretch}{1.5}
\centering
\caption{Sample Quotes}
\label{ChainOfEvidenceTable}
\begin{tabular}{|p{1.65in}|p{5.55in}|}
\hline
\multicolumn{2}{|c|}{Policies} \\
\hline
\textbf{Team Code Ownership} & Example 2a \\
& \participantQuote{I feel ownership of it as a whole and that I feel like I'm empowered and able to like go on and work on any part of the codebase.} \\
& \participantQuote{I don't feel like I have [individual] ownership. It's really a collaborative effort to achieve where we are today \ldots I feel like everybody owns this product.} \\
& \participantQuote{There is a lot of emphasis that you are not your code.} \\
& \participantQuote{I never feel like a specific piece is mine or something belongs to other people.}\\
& \participantQuote{Hey this is our code.} \\
& \participantQuote{I feel the freedom to make changes as I see necessary. It's always a discussion with [my pair]. We come to a conclusion on what we want to do next.} \\
& \participantQuote{I felt like that had my thumbprint on it on the clay.} \\
& \participantQuote{I think we should encourage the feeling of ownership as in I can change anything and this is all mine.} \\ 
& \participantQuote{It is worth having everyone being able to do anything.} \\

\hline
Shared Schedule & Example 2b \\
\hline

Avoid Technical Debt & Example 2c \\
& One participant described his feelings about the code, \participantQuote{I am proud and disgusted by the code.} When asked to explain the disgusted side, the participant said, \participantQuote{We have dug ourselves into technical debt holes several times and then that combined with the time pressure means that like we keep putting it off \ldots I found myself shying away from those parts of the system. I feel less ownership over that part of the code now.} \\
& Caretaking the code by making it better and cleaning up messes. \participantQuote{Sometimes I kind of feel like a janitor to [the code base]. Maybe caretaker would be better. Yeah, probably caretaker. I feel like a janitor just cleans up messes, but a caretaker makes things better.} \\
& \participantQuote{Things that I build can be subsequently removed.} \\
& \participantQuote{Code structure is clean.} \\
& \participantQuote{We are pretty good at separating all the logics in specific classes. I think that our codebase is pretty flexible. When we want to do refactors, it's not super complicated and not super hard to do.} \\
& Thinking \quotes{someone else} will do a refactor or solve a problem means that nobody will do it. \participantQuote{Thinking that someone else will do it can be dangerous, because if everybody thinks that, then nobody will do it.} \\
\hline
\end{tabular}
\end{table*}




\begin{table*}[t]
\renewcommand{\arraystretch}{1.5}
\centering
\caption{Sample Quotes}
\label{ChainOfEvidenceTable}
\begin{tabular}{|p{1.65in}|p{5.55in}|}
\hline
\multicolumn{2}{|c|}{Removing Knowledge Silos Practices} \\
\hline
Continuous Pair Programming &
Example 3a \\
\hline
\textbf{Overlapping Pair Rotation} & Example 3b \\

& \participantQuote{To make sure that knowledge silos don't form we rotate pairs. As people work on specific stories and specific parts of the code, we want to share that knowledge.} \\

& \participantQuote{Rotating pairs reduces knowledge silos and reduces the bus factor. We do not want to the departure of one developer from the project to cripple the project.} \\


& \participantQuote{We rotate pairs because everyone has a different set of knowledge. When you work with someone you get a little bit of that knowledge. The more you pair with them, the more knowledge you get.} \\

& \participantQuote{My ideal team size is two pairs. Anything that we did today, by tomorrow, the team will have an opportunity to know what we've done.}  After one rotation, context is shared immediately throughout the team.} \\

& On one project ūnus, one developer kept the story with everyone else rotating through. After day four, that developer was annoyed that the anchor wanted to refactor \quotes{everything} the team had created. By maintaining continuity on the story, the developer had built strong individual code ownership. \\

& Overlapping Pair Rotation helps spread context through the team. One developer said, 
\participantQuote{I have a general sense of confidence around the codebase. Given any story in the backlog, I'm reasonably confident that I could go and figure it out.} When knowledge silos emerged, we saw reluctance to pick up certain stories. Some developers wanted to only work with the knowledge experts on those stories which then further entrenched the knowledge expert at the expert. A team employing the \textit{Optimizing for context sharing} strategy would prevent one team member from building a knowledge silo. \\

& \participantQuote{At the beginning of the day, I might keep the story or roll off of it and join another story with somebody else already working on it.} \\

& In describing his experience at another Extreme Programming software company, one participant said, \participantQuote{Actually I paired with him for three weeks. I then paired with two weeks [before the holidays] and again throughout the four weeks in January \ldots There were infrequent rotations of pairs except in the event of personality conflicts.} This led to building of deep knowledge silos. The pairs would specifically only work in one part of the code base, and would not be able to easily work in other parts of the code base. \\

& Working on the same track decreases empowering to change anything in the code base. \participantQuote{Yeah, I'd be more hesitant to work on other people's modules. I mean I'd be less empowered to go fix other things. As a result, I feel the code would turn into a lot of very orderly pieces that are connected in terrible ways. I wouldn't feel as much ownership over the whole codebase.} \\

& With a big team, keeping shared context is challenging as so much parallel work is being done. \participantQuote{I feel that we don't have context spread around fully but then again having five, sometimes six pairs on the project makes it go really fast so it's hard to keep context \ldots It is a big team and you can be working on one track for a week perhaps and then the other four pairs move fast. Things just change under you. You get back to some other place and you're saying \quotes{oh what happened here.} Because of that speed, it's harder to keep context on everything. } \\
\hline
Knowledge Pollination & Example 3c \\
\hline
\end{tabular}
\end{table*}




\begin{table*}[t]
\renewcommand{\arraystretch}{1.5}
\centering
\caption{Sample Quotes}
\label{ChainOfEvidenceTable}
\begin{tabular}{|p{1.65in}|p{5.55in}|}
\hline
\multicolumn{2}{|c|}{Caretaking the Code Practices} \\
\hline
TDD / BDD & Example 4a \\
\hline
Continuous Refactoring & Example 4b \\
\hline
& In describing the project's code, the developer said, \participantQuote{It was really easy to do refactoring on the code that we want to modify. So, I guess the code is pretty flexible.} It was desirable for the team to create code that was easy to modify since this allowed future developers to understand and modify the code. \\
& Continuous Refactoring helped developers create components with clear separation of concerns.  One developer said, \participantQuote{[It is easy] to know what objects I need to modify \ldots it really allows me to change things without breaking everything else \ldots they are pretty separated.}  On these projects, Continuous Refactoring helped the developers to know where to find the classes responsible for a particular concern, e.g. the discoverability of code. \\

\hline
Supported by Live on Master & Example 4c \\
&  \participantQuote{We have a preferred way of doing software development that works, things like frequent commits and always rebasing.} \\
& Pivots try to commit and push multiple times a day. In the afternoon, if they have not pushed, they tend to get \quotes{impatient} and begin to look to ways to commit sooner. \\
& On one project, one machine had  multiple commits from several days without integrating to master (almost like a virtual branch). The pair working on that machine asked the team to not change parts of the system since it would result in merge conflicts. Other members of the team inquired about the nature of the work to see if it could be decomposed into incremental refactors. These could be routinely integrated as opposed to one \quotes{big bang} integration. \\
& Several Pivotal Data product teams are open source projects. All of their work is submitted to master via a pull request. This makes everyone work the same way, just like open source contributors outside the organization. The pull request system makes it hard to do refactors. There is a delay from when the code is submitted to when it is merged into master. Since master may have changed, merge conflicts ensue. This discourages the team from doing refactoring and they rarely refactor.  \\
& One engineer described his experience at a previous company that used pull requests. The review often took more than a week. The code changed so much that it no longer was feasible to merge back into master. The engineer would give up with trying to integrate it, use it as a reference implementation, and re-write the change on the current code from scratch. \\
\hline
\end{tabular}
\end{table*}





